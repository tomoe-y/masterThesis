\chapter{結論と展望}\label{chapter6}
2022年よりLHC第三期運転(Run-3)が開始され、第二期運転(Run-2)に比べて重心系エネルギーとルミノシティが増加したことに伴いイベント数が増加する。これらに対応すべく、ATLAS実験のトリガーシステムを改良する必要がある。

トリガーシステムの中でも初段トリガーに分類される初段ミューオントリガーは、ミューオンのヒット情報をもとにCoincidence Windowを参照することで、短時間でミューオンの$p_{\rm{T}}$を算出している。
LHC及びATLAS検出器のアップグレードに伴って、CWをRun-3に対応したものに作り直す必要がある。

以前のCWはシングルミューオンのシミュレーションデータから作成しており、実際の検出器のズレや歪みが考慮されていない。そのため、実際の測定にこのCWを適用するとトリガー性能が低下してしまう。
そこで、検出器のズレや歪みに対する補正を行うことでCWを最適化し、トリガー性能を向上させる手法が先行研究によって確立されたが、検出器の位置ごとに手動で最適化を行うため、作業量が膨大であることが問題となり、従来の手法に代わる効率的なCWの作成及び最適化手法が求められた。
本研究ではエンドキャプ部の初段ミューオントリガーにおいて、トリガー判定の際に使用されているCWを効率よく作成、及び最適化する手法の開発を行った。

本論文では、近年急速に発展している機械学習に注目し、機械学習をCWの作成に用いること作成、及び最適化する作業の効率化を図った。
多層パーセプトロンの機械学習モデルを構築し、シミュレーションデータをトレーニングに使用することで従来の手法で作成したCWと同様の性能を持つCWの作成が可能であることを示した。また、実際のデータをトレーニングに使用することで、CWに対して検出器のズレの補正ができることを証明した。機械学習で作成したCWを用いた際のトリガー性能についての評価を行い、$p_{\rm{T}}$閾値が14~GeVのトリガーに関して、2022年度Run-3で使用されているCWよりも運動量分解能が向上し、トリガー効率が約1$\%$向上したことを確認した。
本研究で開発した手法では実際のデータをトレーニングに用いることで、従来の手法と同等以上の性能のCWを作成できるモデルを学習できだけでなく、TGC検出器の位置による磁場構造の違いや検出器アライメントのズレを自動で織り込んでモデルの学習を行う事ができることを示した。そのため、本研究の手法では、従来のような検出器の位置ごとに手動で最適化を行う必要が無くなるといった利点がある。
また、本研究で提案した作成手法では、機械学習を用いることで従来の手法よりもCWを効率よく短時間で作成することができため、実験期間中に迅速にCWのアップデートが可能となり、今後さらにトリガー性能の向上が期待できる。
さらに、機械学習のトレーニングにRun-3のデータを使用し、CWを作成することで新たな検出器のズレに対応することができ、トリガー性能の向上が期待される。
また、本研究の手法では$p_{\rm{T}}$閾値の変更に対して柔軟な対応が可能である。そのため、今後のアップグレードやトリガーシステムに対する要求の変更に対しても、本研究の手法は即座に対応が可能であると期待される。

検出器の状態や電子回路に起因するような本来では実際に測定した上で補正が必要なものに対して、機械学習を用いることで補正値を自動的に行い、性能を向上させることができることを示した。そのため、本論文で示したTGCのトリガーシステムだけでなく、補正値の測定が難しいNSWやRPCなどの検出器の性能向上に応用が可能である。

