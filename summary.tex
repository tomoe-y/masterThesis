\chapter{結論と展望}\label{chapter6}
2022年よりLHC第3期運転(Run-3)が開始され、Run-2に比べて重心系エネルギーとルミノシティが増加したことに伴いイベント数が増加する。これらに対応すべく、ATLAS実験のトリガーシステムを改良する必要がある。

トリガーシステムの中でも初段トリガーに分類される初段ミューオントリガーは、ミューオンのヒット情報をもとにCWを参照することで、短時間でミューオンの$p_T$を算出している。
LHC及びATLAS検出器のアップグレードに伴って、CWをRun-3に対応したものに作り直す必要がある。

CWはシミュレーションデータから作成しており、実際の検出器のズレや歪みが考慮されていない。そのため、実際の測定にこのCWを適用するとトリガー性能が低下してしまう。
そこで、検出器のズレや歪みに対する補正を行うことでCWを最適化し、トリガー性能を向上させる手法が先行研究によって確立されたが、検出器の位置ごとに手動で最適化を行うため、作業量が膨大であることが問題となり、従来の手法に代わる効率的なCWの作成及び最適化手法が求められた。
本研究では、エンドキャプ部初段ミューオントリガーにおいてトリガー判定の際に使用されているCWを効率よく作成及び最適化する手法の開発を行った。
本論文では、多層パーセプトロンの機械学習モデルを構築し、実際のデータをトレーニングに使用することでCWの作成が可能であることを示した。さらに本研究で開発した手法では、実際のデータをトレーニングに用いたことで、TGC 検出器内の位置による磁場構造の違いや検出器アライメントのずれを自動で織り込んでモデルの学習を行う事ができ、従来のような検出器の位置ごとに手動で最適化を行う必要がなくなる利点がある。
また、このCWを用いた際のトリガー性能についての評価を行い、2022年度Run-3で使用されているCWよりも運動量分解能が向上したことを示した。さらに、Run-3に導入した際のトリガー性能の見積もりを行い、トリガー性能が向上することを示した。

本研究で提案するCWの作成手法では、機械学習を用いることでCWを効率よく短時間で作成することができる。そのため、実験期間中に迅速にCWのアップデートが可能となり、今後さらにトリガー性能の向上が期待される。また、機械学習のトレーニングにRun-3のデータを使用し、CWを作成することでさらに検出器のズレに対応することができ、トリガー性能の向上が期待される。



