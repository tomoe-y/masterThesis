\chapter{結論}\label{chapter6}
ATLAS実験では2022年から世界最高の重心系エネルギー~($\sqrt{s}=\SI{13.6}{\TeV}$)で第3期運転~(Run-3)が開始された。
本研究では~Run-3から新たに導入されたトリガーのうち、以下の2種類のミューオントリガーについて~Run-3実データを用いて評価を行った。

\subsubsection{近接2ミューオンのためのトリガー}
Run-2まで2ミューオンが近接している場合、2ミューオントリガーのトリガー効率が低下してしまうという問題があった。
トリガーの非効率に対処するため、L1、HLTともに先行研究\cite{article:taniguchi}において改良が行われ、Run-3から実装された。

改良されたアルゴリズムを~Run-3実データに対して用いた場合のトリガー効率を評価した結果、L1、HLTともに2つのミューオン同士が非常に近接している$\Delta R<0.15$の領域でトリガー効率が向上していることを確認した。
またこの結果はシミュレーションでよく再現され、想定通りに動作していることが確認された。

\subsubsection{NSWを用いた~L2MuonSA}
LHCのルミノシティ増加に伴う検出器へのヒットレート増加に対応するために~SWに代わって~NSWが~Run-3から新たに導入された。
NSWは~sTGC8層と~MM8層の合計16層で構成され、ミューオン検出器の最内層であるインナーステーションに設置されている。

NSWを用いた~L2MuonSAアルゴリズムを~Run-3実データに対して用いた場合の$\pt$分解能の評価を行った結果、L2MuonSAで~NSWを用いた時の$\pt$分解能がシミュレーションから想定されていたよりも悪いということがわかった。
また~L2MuonSAでインナーステーションである~NSWの情報を用いずにミドルステーションとアウターステーションの情報のみを用いた場合の方が、NSWを用いた場合と比べて$\pt$分解能がよいことがわかった。
その原因として部分飛跡再構成に用いる~MMのヒット数が少ないことが考えられた。
L2MuonSAで部分飛跡を再構成する際に~sTGC、MMそれぞれ独立にヒット選択アルゴリズムを用いて再構成に用いるヒットの選択を行う。
このアルゴリズムはヒットが8層の内かなりの層で直線的に並んでいるヒットがあることを要求するが、実データでは~MMのヒットは直線的に並んでいるイベントは少なかった。一方で~sTGCは想定通りヒットを選ぶことができていた。

そこで従来のアルゴリズムで選択された~sTGCのヒットを用いて~MMのヒットを選ぶアルゴリズムを新たに検討した。
実データに対して用いた結果、部分飛跡再構成に用いる~MMのヒットの数は増えたが、従来のアルゴリズムでの$\pt$分解能から改善されなかった。
MMの分解能が~sTGCに比べて悪いのが原因であると考えられる。
これらの結果から、現在用いられているアルゴリズムは明らかな改善の余地があるわけではなく、ほぼ最適に動作していると結論した。
また、L2MuonSAアルゴリズムに~NSWを用いるには分解能の向上など~NSW検出器の理解と動作改善をより進める必要がある。