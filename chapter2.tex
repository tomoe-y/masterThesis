\chapter{LHC-ATLAS実験}
\label{chapter2}

\section{LHC加速器}
\label{section2-1}

Large Hadron Collider (LHC)は、スイスのジュネーブ郊外にある欧州素粒子原子核研究機構 (CERN)の地下に建設された周長約27kmの世界最大の加速器である。LHC加速器では高いエネルギーで陽子を衝突させることにより、


\section{ATLAS実験}
\label{section2-2}
\subsection{ATLAS検出器}
ATLAS検出器は、LHCの衝突点の1つに設置された、直径25m、長さ44mの円筒形の大型汎用検出器である。
ATLAS検出器は複数の検出器を組み合わせて構成されており、内側から内部飛跡検出器、カロリメータ、ミューオンスペクトロメータといった検出器が設置されている。また、内部飛跡検出器とカロリメータの間には超電導ソレノイド磁石、カロリメータの外側にはトロイド磁石がそれぞれ設置されており、磁場によって曲げられた荷電粒子の曲率半径から運動量の算出を行っている。


\subsection{ATLAS検出器における座標系}
ATLAS検出器における座標系を示す。

\begin{figure}[tb]
  \centering
  \includegraphics[clip, width=14cm]{fig/2/atlas_coordinate_fix.pdf}
  \caption{ATLAS検出器における座標系}
  \label{fig:a}
\end{figure}

衝突点を原点とし、ビーム軸に沿ってz軸を取る。地面に水平方向にx軸を取り、x-z平面に垂直方向にy軸を取る。

\subsection{内部飛跡検出器}

\subsection{カロリメータ}

\subsection{ミューオンスペクトロメータ}
\label{section2-2-4}

\subsubsection{・Thin Gap Chambers (TGC)}



\subsection{マグネットシステム}
\subsubsection{・ソレノイド磁石}




