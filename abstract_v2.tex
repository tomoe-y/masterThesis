\pagestyle{empty}

\begin{center}
    \begin{large}
        令和5年度~修士論文発表会概要 
    \end{large}
\end{center}

\begin{center}
  \begin{Large}
     LHC-ATLAS実験Run-3における\\新ミューオントリガーアルゴリズムの動作検証と改良
  \end{Large}
\end{center}

\begin{flushright}
    粒子物理学研究室~220s126s~山下智愛 \\
    指導教員~山\ajTatsuSaki 祐司
\end{flushright}

\vspace{10pt}

Large~Hadron~Collider~(LHC)は欧州原子核機構~(CERN)によってスイス・ジュネーブの地下に設置された世界最高エネルギーの陽子陽子衝突型加速器である。ATLAS実験は、LHCの陽子陽子衝突点の1つに大型汎用検出器を設置し、新粒子探索や標準理論の精密測定まで幅広い物理を研究対象としている。

LHCの陽子陽子衝突頻度は40MHzで、すべての衝突事象に対して処理を行い記録することは不可能である。そのためATLAS実験では、トリガーシステムを用いて興味のある事象のみを選別し取得することによって、データ取得レートの削減を行っている。
本研究で扱うミューオントリガーは、ハードウェアによる高速な判定を行う初段トリガー、ソフトウェアを用いて高速演算を行い飛跡を再構成する後段トリガーの2段階である。

LHCでは2022年から重心系エネルギー~($\sqrt{s}=\SI{13.6}{\TeV}$)で第三期運転~(Run-3)が開始された。
ルミノシティ向上により物理事象のデータをより多く得ることができる一方で、1回のバンチ衝突における多重反応(パイルアップ)や背景事象の増加に伴い検出器へのヒットレートが増加する。
これに対応するために~ATLAS実験では~SWに代わって~NSW~(New~Small~Wheel)を新たに設置した。
加えて、2ミューオン事象ではトリガーに2つのミューオンを要求することで2ミューオン事象のレートを抑えることができるが、2つのミューオン同士が近接している場合2ミューオントリガーのトリガー効率が低下してしまうことが~Run-2まで問題になっていた。


本研究では、Run-3から新たに加えた近接2ミューオンのためのトリガーについて、実データを用いて初段トリガーと後段トリガーのそれぞれについて~Run-3実データを用いて動作検証を行った。
その結果、以前のトリガーでは取れていなかった2ミューオンが非常に近接している領域で、効率の回復を確認した。

また、NSWを用いた後段ミューオントリガーのアルゴリズムの動作検証を行った。モンテカルロシミュレーションの結果をもとに、NSWを用いると運動量分解能がよくなると想定されていたが、実際には改善しなかった。
本論文ではその原因を調べ、改良を試みた。