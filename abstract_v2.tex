\begin{center}
\begin{huge}
概要
\end{huge}
\end{center}

\vspace{10pt}

Large~Hadron~Collider~(LHC)は、欧州原子核研究機構~(CERN)に建設された世界最高エネルギーの陽子陽子衝突型加速器である。ATLAS検出器は、LHCの衝突点の1つに設置された大型汎用検出器であり、TeVエネルギー領域の陽子衝突で生成される粒子を測定することで、新粒子の直接探索やヒッグス粒子の精密測定を行い、素粒子標準模型を超えた新物理の発見を目指している。

LHCにおける陽子陽子衝突は高頻度であるため、すべての衝突事象を記録することはできない。そのためトリガーシステムを使用し、物理として興味のある事象のみを選別し保存している。
LHC及びATLAS検出器は2019年から2022年までの期間にアップグレードが行われ、2022年7月から重心系エネルギー13.6~TeVでの第三期運転~(Run-3)として実験が行われている。

ルミノシティの増加に伴って事象数も増加するため、物理事象の選別を行うトリガーシステムの改良が必要不可欠である。
トリガーシステムの中でも初段トリガーに分類されるエンドキャプ部の初段ミューオントリガーは、Thin-Gap~Chambers~(TGC)というミューオン検出器を用いて横方向運動量を指針としたトリガー判定を行っている。このとき、ミューオンの飛跡の曲がり具合と横方向運動量の対応を関連づけたCoincidence~Window~(CW)と呼ばれる参照表を用いることで、短時間で飛跡情報からミューオンの横方向運動量を算出している。

初段ミューオントリガーのトリガー性能の改善に向けて、CWをRun-3に対応したものに作り直す必要がある。
CWはシミュレーションデータから作成しているため、実際の検出器のズレや歪みが考慮されておらず、そのまま実際の測定に使用するとトリガー性能が低下してしまう。
そこで、検出器のズレや歪みに対する補正を行うことでCWを最適化し、トリガー性能を向上させる手法が先行研究によって確立されてきた。
しかし、従来の手法ではCWの作成の手間の多さや手動で行う最適化の作業量が膨大であることが問題となり、従来の手法に代わる効率的なCWの作成及び最適化手法が求められた。そこで、機械学習を用いて効率的にCWを作成し最適化を行う手法の開発を行う。

本研究では、実際の実験データを機械学習のトレーニングに使用することで、自動的に検出器のズレや歪みに対する補正を行うことのできる手法の開発を行い、CW作成の効率化を図った。また、作成したCWを用いた際のトリガー性能について、従来の手法で作成したCWと比較を行い、本研究の手法によってトリガー性能を向上させることが可能であることを示した。本発表では、開発したCWの作成手法と、作成したCWのトリガー性能の評価について述べる。