\chapter{序論}
我々の身の回りにある物質を構成する最小単位は素粒子である。
この物質の最小単位である素粒子と、素粒子の相互作用を記述した理論として標準模型が存在している。素粒子物理学において、自然界には電磁相互作用、強い相互作用、弱い相互作用、重力相互作用といった4種類の基本相互作用が存在すると考えられており、標準模型では重力相互作用以外の3種類の相互作用が記述されている。
標準模型は、12種類のフェルミオン、4種類のゲージボソン、ヒッグス粒子の計17種類の粒子から構成されている。
2012年にヒッグス粒子がATLAS実験とCMS実験によって実験的に発見されたことによって、標準模型は。
しかし、標準模型には多くの未解決問題が残されており、この問題を解決するためには、標準模型を超える新しい物理が必要であり、新しい物理を探索するための実験が世界中で行われています。


