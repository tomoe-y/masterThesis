\chapter{序論}
我々の身の回りにある物質を構成する最小単位は素粒子である。
物質の最小単位である素粒子と、素粒子の相互作用を記述した理論として標準模型が存在している。素粒子物理学において、自然界には電磁相互作用、強い相互作用、弱い相互作用、重力相互作用といった4種類の基本相互作用が存在すると考えられており、標準模型では重力相互作用以外の3種類の相互作用が記述されている。
標準模型は図\ref{fig:標準模型}に示すように、12種類のフェルミオン、4種類のゲージボソン、ヒッグス粒子の計17種類の粒子から構成されており、2012年に唯一実験的に未確認であったヒッグス粒子が発見された。
\begin{figure}[tb]
  \centering
  \includegraphics[clip, width=10cm]{fig/1/standardmodel.jpg}
  \caption{標準模型を構成する17種類の素粒子}
  \label{fig:標準模型}
\end{figure}

現在、標準模型は多くの物理事象を説明することができているが、ダークマターの存在や階層問題などの多くの未解決問題が残されている。
この問題を解決するためには、標準模型を超える新しい物理が必要であり、この新物理を探索するために世界中で様々なアプローチの実験が行われている。

このアプローチの一つとして、ジュネーブ郊外に位置する欧州原子核研究機構 (CERN) の地下に設置された Large Hadron Collider (LHC) を用いる高エネルギーの陽子陽子衝突実験がある。LHC を使った衝突実験の1つである ATLAS 実験では、ATLAS 検出器と呼ばれる大型汎用検出器を用いて陽子-陽子衝突によって生成される粒子を検出し、TeV スケールの物理事象までの測定を行い標準理論を構成する素粒子の精密測定や超対称性粒子の探索を目的としている。

ATLAS 実験では陽子を$40$ MHzの頻度で衝突させたデータを用いて新物理探索を行っている。しかし、計算機リソースやデータ容量などの観点からこの高頻度での陽子-陽子衝突事象を全て保存することができない。そのため、トリガーシステムを使った事象選別を行うことで膨大な量のデータから物理として興味のある事象を選別し、保存可能なデータ量まで事象を減らして保存している。
LHC及びATLAS 検出器は2018年から2021年までの期間にアップグレードが行われ、2022年からRun-3として運転が再開された。アップグレードを行ったことで、陽子-陽子衝突の重心系エネルギーや瞬間ルミノシティが増加する。そのため、Run-3の運転で増加したイベントレートや今後予定されている高輝度運転に対応したより効率の良いトリガーシステムの実装が必須である。
ATLAS 検出器では、1段目にはハードウェアベースの高速処理が可能な初段トリガー、2段目ではソフトウェアベースで精密処理が可能な後段トリガーのような2段階のトリガーシステムが実装されている。
本研究では、この初段トリガーの一つであるミューオンをターゲットにして事象選別を行う初段エンドキャプ部ミューオントリガーの改良を行った。

初段エンドキャプ部ミューオントリガーでは Thin Gap Chambers (TGC) というミューオン検出器を用いてミューオンの測定を行い、ミューオンのヒット情報から算出した運動量を指針としてトリガー判定を行っている。
このとき、 Coincidence Window (CW) と呼ばれるミューオンの飛跡の曲がり具合と運動量の対応を関連づけた参照表を用いてミューオンの飛跡情報から運動量を算出している。この CW はシミュレーションサンプルをもとに作成されているため、実際の検出器のズレや歪みが考慮されていない。そのため、CW に対して実際の検出器のズレや歪みの補正を行うことでトリガー性能の向上を図っている。

以前の研究により、CW に対する検出器のズレや歪みの補正方法が確立されているが、作業量が膨大であるために検出器やシステムがアップデートされた Run-3 の CW に対して以前の補正手法を行うことは難しい。
そこで、近年発展が著しい機械学習に着目し、本研究では機械学習を使うことで検出器アライメントも含めた CW の作成手法の開発を行う。



