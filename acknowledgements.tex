\chapter*{謝辞}
\addcontentsline{toc}{chapter}{謝辞}

本論文の作成にあたり、多くの方々にご指導ご鞭撻を賜りました。

指導教員である前田順平先生には、非常にお世話になりました。深く御礼申し上げます。何もわからなかった私にATLAS実験のことや、ソフトウェア、機械学習といった最新技術などの知識を、根気よく指導いただいたおかげで研究を進めていくことができました。また、CERNに行く機会をいただき、心から感謝しております。実際にCERNに行き、ATLAS検出器を目にすることができたことはとても良い経験になりました。学会等の発表準備の際には何度も指導や添削をして頂き、本当にありがとうございました。修士課程の2年間で大変多くの知識と技術を得ることができたのは前田先生のおかげです。

神戸ATLASグループの藏重久弥先生、山\ajTatsuSaki 祐司先生、越智敦彦先生には、神大内部のミーティングや日常の研究生活において様々な助言をいただきましたこと、大変感謝しております。
また、竹内康雄先生、身内賢太朗先生、鈴木州先生に、東野聡先生には研究室ミーティングのコロキウムで様々なことをご指導いただきました。お礼申し上げます。

神戸大学 ATLAS グループの先輩である日比宏明さん、安部草太さん、寺村七都さん、池森隆太郎さん、野口健太さんには多くのことで手助けをしていただきました。
研究の技術的な部分の質問にも懇切丁寧にご教示頂いたことで、本研究をやり遂げることができました。本当にありがとうございました。
研究以外の面でも多くの面倒を見ていただいたことは心の大きな支えでした。色々とお世話になりました。ありがとうございます。

研究室の同期である丸元星弥君、金崎奎君、中山郁香さん、山下翼君、高橋真斗君、みんなと切磋琢磨し、互いに励ましあったことで、ここまで研究をやり遂げることがで
きました。みんなのおかげでとても充実した研究生活を送ることができました。
丸元君、何度も研究室で徹夜しましたね。今となってはいい思い出です。金崎君、一緒に帰宅しながらいろんな相談に乗ってくれてありがとう。中山さん、卒業研究では色々お世話になりました。とても助かりました。山下君、よく雑な絡み方をして困らせてしまいましたが、私の相手をしてくれてありがとう。高橋君、研究の相談から日常の雑談まで多くのことをこの2年間語り合いましたね。君との雑談の時間はとても良い気分転換になり、研究生活の心の支えでした。ありがとう。博士課程でも頑張ってください。
後輩の田路君、山下さん、森本君、先輩としてあまり大層なことはできなかったですが、君たちのより一層の活躍を願っています。


また、ATLAS Japanグループの皆様には大変お世話になりました。KEKの青木雅人先生、東京大学の斉藤智之先生、名古屋大学の堀井泰之先生及びATLAS JapanのTGCグループに所属している先輩方には日頃よりお世話になりました。ミーティングでは多くの助言を頂き、CERN滞在中には気にかけていただきありがとうございました。

最後に、これまで何不自由のなく育ててくれた家族に感謝の意を表して、謝辞とします。





