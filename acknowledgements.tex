\chapter*{謝辞}
\addcontentsline{toc}{chapter}{謝辞}

本論文の作成にあたり、多くの方々にご指導ご鞭撻を賜りました。

指導教員である山\ajTatsuSaki 祐司先生には、学部の卒業研究から3年間指導していただき本当にお世話になりました。
ATLAS実験のことやソフトウェアについてだけでなく、研究の進め方やミーティングや学会での話し方など多くの面でご指導いただきました。
わからないことがあるときや研究に行き詰ったときはすぐに尋ねてしまいましたが、いつでも真摯に対応してくださりわからないことは一緒に論文を読んで考えてくださったおかげで一歩ずつ研究を進めることができました。
学会前や修士論文提出前などギリギリになって焦りだす私に対しても根気強く指導していただき、お忙しい中多くの時間をとってオンラインやオフラインで何度もミーティングしていただき本当に助かりました。
最後まで手のかかる学生だったと思いますが、ご指導いただけたこと心から感謝しております。
3年間で多くの知識と技術を得ることができたのは山\ajTatsuSaki 先生のおかげです。本当にありがとうございました。

神戸~ATLASグループの藏重久弥先生、越智敦彦先生、前田先生には神戸大学内部の~ATLASミーティングで多くのご指導をしていただきました。
特に前田先生には~L1について取り組む際に、研究の方針や技術的なことなど様々なアドバイスをたくさん頂きました。学会前切羽詰まって質問に伺った際、研究のアドバイスに加えてお菓子を頂けたことは忘れません。ありがとうございました。

神戸大学粒子物理学研究室の、竹内康雄先生、身内賢太郎先生、鈴木州先生には研究室内でのコロキウム等でご助言を頂きました。ありがとうございました。

ATLAS~Japanグループの皆様には大変お世話になりました。
HLTグループの長谷川庸司先生、長野邦浩先生、山口洋平先生には、HLTや~NSWについて教えていただいたりミーティングでコメントを頂くなど大変お世話になりました。
特に山口先生にはコンピューティングの技術的なことや日々の研究の方針などたくさんの相談に乗っていただきました。日中でも深夜でもすぐに対応していただき、すごく心強かったです。ありがとうございました。

東京大学の古川真林さんにはCERN出張の際に大変お世話になりました。
初対面の私のために様々な準備をしていただき、生活面でとても助けていただきました。
ほとんど知り合いがいない状態での出張で不安でしたが古川さんがいてくれて心強かったです。CERN出張を楽しめたのは古川さんのおかげといっても過言ではないと思います。本当にありがとうございました。

研究室の同期である、大藤瑞乃さん、田路航也くん、森本晴己くん、高木優祐くんのおかげで楽しい学生生活を送ることができました。
私は研究室の同期の皆さんのことがとても好きでした。いつもたくさん話しかけてしまい、迷惑だったかなと少しだけ反省しております。
大藤さん、そのパワフルさにいつも元気をもらっていました。
田路くん、隣の席ということもあり特にたくさん話しかけてしまいましたが、いつも相手してくれてありがとうございました。
森本くん、CERN滞在楽しかったですね。前歯きれいになってよかったです。
高木くん、いつも神岡にいて神戸で一緒に研究することは少なかったですが、どんな状況でも頑張ろうとする姿に励まされていました。
とてもとても個性的なメンバーで学年としてまとまりはなかったですが、この2年間一緒に研究をして学会や研究会など遠足に行くことができて本当に楽しかったです。
研究を進める中で同期の存在は支えになっていました。
ありがとうございました。

神戸~ATLASグループの後輩である村田優衣さん、水引龍吾くん、西将汰くん、樋口流雲くん、張力くんはみんなで切磋琢磨しながら研究に取り組んでいて、その姿勢に刺激を受けていました。
お菓子をくれたりダジャレで笑わせてくれたりいつも本当に元気をもらっていました。ありがとうございました。

最後に私のやりたいことを否定せず、尊重し応援してくれた家族には感謝してもしきれません。
とても反抗期が長く本当に大変だったと思いますが、ここまで育ててくれてありがとうございました。
これからは親孝行ををたくさんしようと思っているので長生きしてください。これからもよろしくお願いします。