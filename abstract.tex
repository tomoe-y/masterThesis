\begin{center}
  \begin{huge}
    概要
  \end{huge}
\end{center}

\vspace{10pt}

Large Hadron Collider (LHC) は、スイスのジュネーブ近郊に位置する欧州原子核研究機構 (CERN) に建設された世界最高エネルギーの陽子陽子衝突型加速器である。ATLAS 検出器は、LHC の衝突点の 1 つに設置された大型汎用検出器であり、TeV エネルギー領域での陽子衝突で生成される粒子を測定することで新粒子の直接探索やヒッグス粒子の精密測定を行い、素粒子標準理論を超えた新物理の発見を目指している。

LHC 及び ATLAS 検出器は 2018 年から 2022 年までアップグレードが行われ、2022 年から第三期運転 (Run-3) として実験が開始された。Run-3 では重心系エネルギー $13.6 TeV$、瞬間最高ルミノシティ $2.0\times10^{34} cm^{−2}s^{−1}$ での運転が計画されている。

LHC における陽子陽子衝突の頻度(イベントレート)は 40 MHz だが、データ記録容量の制限のためにすべての衝突事象を記録することはできない。そのため、トリガーシステムを使用して物理として興味のある事象を選別し、必要な事象のみを取得する必要がある。ATLAS 実験で使用されているトリガーシステムは大きく分けて 2 段階に分けられる。1段目のハードウェアベースの高速処理が可能な初段トリガーでは、全事象に対しトリガー判定を行い 100 kHz 以下までイベントレートを落とす。そして、2段目のソフトウェアベースの後段トリガーで精密なトリガー判定を行い、数kHzまでイベントレートを落としたデータを保存する。

ATLAS 実験は 2022 年度の運転において一時的に瞬間ルミノシティ $2.5\times10^{34} cm^{−2}s^{−1}$ まで到達したことを受け、2023 年以降の運転では計画より高いルミノシティでの運転が予定されている。ルミノシティの増加に伴ってイベントレートも増加するため、保存すべき重要な物理事象の選別を行うトリガーシステムの改良が必要不可欠である。

トリガーシステムの中でも初段トリガーに分類される初段ミューオントリガーでは Thin Gap Chambers (TGC) というミューオン検出器を用いてミューオンの測定を行い、ミューオンのヒット情報から算出した横方向運動量を指針としてトリガー判定を行っている。
このとき、ミューオンの飛跡の曲がり具合と横方向運動量の対応を関連づけた Coincidence Window (CW) と呼ばれる参照表を用いることで、短時間で飛跡情報からミューオンの横方向運動量を算出している。この CW は従来の手法ではシミュレーションデータから作成する。そのため、シミュレーション上には無い実際の検出器のズレや歪みが考慮されていないために、実際の測定にこの CW 適用するとトリガー性能が低下してしまう。
そこで、検出器のズレや歪みに対する補正を行うことで CW を最適化し、トリガー性能を向上させる手法が先行研究によって確立された。
LHC 及び ATLAS 検出器のアップグレードに伴って初段ミューオントリガーに関しても新検出器の導入や電子回路の改良が行わた。そこで CW を Run-3 に対応したものに作り直す必要があるが、従来の CW の作成及び最適化の手法は作成の手順や補正の作業量が膨大であることが問題であり、従来の手法に代わる効率的な CW の作成及び最適化手法が求められた。

そこで、本研究では実際のデータを学習に使用した機械学習を用いて CW を作成する手法を開発した。

<以下結果を書く>

