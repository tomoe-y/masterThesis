\begin{center}
  \begin{huge}
    概要
  \end{huge}
\end{center}

\vspace{10pt}










初段エンドキャプ部ミューオントリガーでは Thin Gap Chambers (TGC) というミューオン検出器を用いてミューオンの測定を行い、ミューオンのヒット情報から算出した運動量を指針としてトリガー判定を行っている。
このとき、 Coincidence Window (CW) と呼ばれるミューオンの飛跡の曲がり具合と運動量の対応を関連づけた参照表を用いてミューオンの飛跡情報から運動量を算出している。この CW はシミュレーションサンプルをもとに作成されているため、実際の検出器のズレや歪みが考慮されていない。そのため、CW に対して実際の検出器のズレや歪みの補正を行うことでトリガー性能の向上を図っている。

以前の研究により、CW に対する検出器のズレや歪みの補正方法が確立されているが、作業量が膨大であるために検出器やシステムがアップデートされた Run-3 の CW に対して以前の補正手法を行うことは難しい。
そこで、近年発展が著しい機械学習に着目し、本研究では機械学習を使うことで検出器アライメントも含めた CW の作成手法の開発を行う。