\begin{center}
  \begin{huge}
    概要
  \end{huge}
\end{center}

\vspace{10pt}

Large~Hadron~Collider~(LHC)は欧州原子核機構~(CERN)によってスイス・ジュネーブの地下に設置された世界最高エネルギーの陽子陽子衝突型加速器である。ATLAS実験は、LHCの陽子陽子衝突点の1つに大型汎用検出器を設置し、新粒子探索や標準理論の精密測定まで幅広い物理を研究対象としている。

LHCの陽子陽子衝突頻度は40MHz、すべての衝突事象に対して処理を行い記録することは不可能である。そのためATLAS実験では、トリガーシステムを用いて膨大のデータの中から興味のある事象のみを選別し取得することによって、データ取得レートの削減を行っている。
本研究で扱うミューオントリガーは、ハードウェアによる高速な判定を行う初段トリガー、ソフトウェアを用いて高速演算を行い飛跡を再構成する後段トリガーの2段階である。測定したミューオンの横運動量~($p_{T}$)に対して閾値を定義することで高い$p_{T}$を持つミューオンを含む事象を選択している。

LHC及びATLAS検出器では、2019年から2022年までのシャットダウンの間にアップグレードが行われ、2022年7月から重心系エネルギー~($\sqrt{s}=13.6$)TeVで第三期運転~(Run-3)としてデータ取得が開始した。
LHCでは、シャットダウン前よりもルミノシティを向上することを目的として加速器の改良が行われた。ルミノシティ向上により物理事象のデータをより多く得ることができる一方で、一回のバンチ衝突における多重反応(パイルアップ)や背景事象の増加に伴い検出器へのヒットレートが増加する。ミューオン検出器において最内層に従来設置されていた~SW~(SmallWheel)では増加したヒットレートに耐えることができず、トラッキング性能が低下する。これに対応するためにATLAS実験ではSWに代わってNSW(NewSmallWheel)を新たに設置した。

本研究では、Run-3から新たに加えた近接2ミューオンのためのトリガーについて2022年に測定されたデータを用いて初段トリガーと後段トリガーのそれぞれについて動作検証を行った。
Run-3以前は検出器で2つのミューオン同士が非常に近接している場合、トリガーの効率が低下するという問題があった。この問題に対処するために初段ではミューオンが近接していることを判定できるトリガーの検証を行った。
その結果、以前のトリガーでは取れていなかった2ミューオンが非常に近接している領域で効率が回復していた。

また、NSWを用いた後段ミューオントリガーのアルゴリズムの動作検証を行った。シミュレーションの結果をもとに、NSWを用いると運動量分解能がよくなると想定されていたが、実際にはNSWを用いると悪化することがわかった。
本論文ではその原因を調べ、改良を試みた。