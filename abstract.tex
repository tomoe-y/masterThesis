\begin{center}
  \begin{huge}
    概要
  \end{huge}
\end{center}

\vspace{10pt}

Large Hadron Collider~(LHC) は、欧州原子核研究機構~(CERN) に建設された世界最高エネルギーの陽子陽子衝突型加速器である。ATLAS検出器は、LHCの衝突点の1つに設置された大型汎用検出器であり、TeVエネルギー領域の陽子衝突で生成される粒子を測定することで、新粒子の直接探索やヒッグス粒子の精密測定を行い、素粒子標準模型を超えた新物理の発見を目指している。

LHCにおける陽子陽子衝突は高頻度であるため、すべての衝突事象を記録することはできない。そのためトリガーシステムを使用し、物理として興味のある事象のみを選別し保存している。
LHC 及び ATLAS 検出器は2019年から2022年までの期間にアップグレードが行われ、2022年7月から重心系エネルギー$13.6~{\rm TeV}$での第三期運転~(Run-3)として実験が開始された。

ルミノシティの増加に伴ってイベントレートも増加するため、物理事象の選別を行うトリガーシステムの改良が必要不可欠である。
%LHCにおける陽子陽子衝突の頻度は40~MHzだが、データ記録容量の制限のためにすべての衝突事象を記録することはできない。そのため、トリガーシステムを使用して物理として興味のある事象を選別し、必要な事象のみを取得する必要がある。ATLAS実験で使用されているトリガーシステムは高速処理が可能な初段トリガーと精密なトリガー判定が可能な後段トリガーの2段階に分けられる。
トリガーシステムの中でも初段トリガーに分類される初段ミューオントリガーは、Thin-Gap Chambers~(TGC)というミューオン検出器を用いて横方向運動量を指針としたトリガー判定を行っている。
このとき、ミューオンの飛跡の曲がり具合と横方向運動量の対応を関連づけたCoincidence Window~(CW)と呼ばれる参照表を用いることで、短時間で飛跡情報からミューオンの横方向運動量を算出している。

%検出器のアップグレードに伴って、CWをRun-3に対応したものに作り直す必要がある。
%従来の手法では、CWをシミュレーションデータから作成し、実際のデータを用いてシミュレーション上にはない実際の検出器のズレや歪みに対する補正を手動で行っていた。
LHC及びATLAS検出器のアップグレードに伴って新検出器の導入や電子回路の改良が行われ、CWをRun-3に対応したものに作り直す必要がある。
CWはシミュレーションデータから作成されるため、実際のデータを用いてシミュレーション上にはない実際の検出器のズレや歪みに対する補正を行う。
そのために検出器のズレや歪みが考慮されず、実際の測定に使用するとトリガー性能が低下してしまう。
そこで、検出器のズレや歪みに対する補正を行うことでCWを最適化し、トリガー性能を向上させる手法が先行研究によって確立されてきた。
LHC及びATLAS検出器のアップグレードに伴って新検出器の導入や電子回路の改良が行われ、CWをRun-3に対応したものに作り直す必要がある。
しかし、従来のCWの作成及び最適化の手法は、作成の手順や、補正の作業量が膨大なため、従来の手法に代わる効率的なCWの作成及び最適化手法が求められた。

そこで、本研究では機械学習を用いて効率的にCWを作成し最適化を行う手法の開発を行う。
本手法では実際の実験データを機械学習のトレーニングに使用し、検出器のズレを直接CWに反映させることが可能である。
本手法で作成したCWを用いた際のトリガー性能についての評価を行い、トリガー効率が向上することを確認した。





%1段目のハードウェアベースの高速処理が可能な初段トリガーでは、全事象に対しトリガー判定を行い 100~kHz以下まで事象を削減する。そして、2段目のソフトウェアベースの後段トリガーで精密なトリガー判定を行い、数kHzまでイベントレートを落としたデータを保存する。
%ATLAS実験は2022年度の運転において一時的に瞬間ルミノシティ$2.5\times10^{34}~{\rm cm^{-2}s^{-1}}$まで到達したことを受け、2023年以降の運転では計画より高いルミノシティでの運転が予定されている。ルミノシティの増加に伴ってイベントレートも増加するため、保存すべき重要な物理事象の選別を行うトリガーシステムの改良が必要不可欠である。
%近年様々な分野で活用されている機械学習は、膨大なデータから特徴や傾向を自動的に読み取り予測を行うデータ分析の方法である。
