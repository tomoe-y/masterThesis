\chapter{機械学習を用いたCoincidence Windowの作成手法}\label{chapter4}

\section{機械学習}\label{回帰分析}

\subsection{ニューラルネットワーク}

\subsubsection{多層パーセプトロン}

\section{機械学習を用いたCW作成手法}

\subsection{入力データに対する事前処理}\label{事前処理}

\subsubsection{TGCの位置情報におけるナンバリングの変換}

\subsubsection{磁場構造を考慮した学習領域の分割}

\subsubsection{ミューオン情報の選別}

\subsection{機械学習モデルの設計方法とトレーニング}

\subsubsection{機械学習モデルの設計}

\subsubsection{ハイパーパラメータ}

\subsubsection{トレーニング}

\subsubsection{機械学習モデルの性能評価}

\subsection{出力データを\texorpdfstring{$p_{\rm{T}}$}{TEXT}閾値に変換}
\subsubsection{トリガー効率の算出}

\subsubsection{フィッティング関数の定義}\label{section:fitting}

\subsubsection{15段階\texorpdfstring{$p_{\rm{T}}$}{TEXT}閾値への変換}