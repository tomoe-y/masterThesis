\chapter{機械学習を用いた Coincidence Window の作成手法}
本章では機械学習を用いた CW の作成手法について述べる。
本研究では、シミュレーションで用いるトリガー用の CW と実際の測定で用いるトリガー用の CW をそれぞれ作成している。

\section{作成手法の方針}
作成手法の方針について述べる。
\ref{section:CW}節で述べたように、初段ミューオントリガーではミューオンの飛跡の曲がり具合を$\Delta R$,$\Delta \phi$として測定し、ヒット位置における CW を参照することでミューオンの運動量を算出している。


そこで、膨大な統計を活用できる手法として近年発展が著しい機械学習に着目する。機械学習の中でも、回帰分類を行うことのできるMultilayer perceptron (MLP) に注目する。
本研究では機械学習を用いて CW を作成することを目的とする。

MC Dataを分けて学習\\

\begin{figure}[tb]
  \centering
  \rule{8cm}{6cm}
  \caption{本研究の方針}
  \label{fig:fit_def}
\end{figure}

\section{機械学習}
機械学習とは、データからコンピュータが自動で特徴量やルールを学習し、学習した結果に基づいて新たなデータに対し分類や予測を行う分析手法の一つである。
そのため、人がコンピュータにルールを明示的に与える代わりに、学習するためのデータを与えることでコンピュータが自動的にデータからルールを獲得することができる。


\begin{figure}[tb]
  \centering
  \rule{8cm}{6cm}
  \caption{回帰、分類の図}
  \label{fig:fit_def}
\end{figure}

\subsubsection{・回帰問題}
回帰の主な目的は、連続値などの値の予測である。

\subsubsection{・分類問題}
分類の目的は、入力されたデータを複数のクラスに分類することである。



\subsection{学習手法}
機械学習には、教師あり学習、教師なし学習、強化学習といった代表的な学習手法がある。
\subsubsection{・教師あり学習}
教師あり学習の目標は、入力$x$から推定したい出力$y$を予測できるようなモデル$y = f(x;\theta)$を学習することである。
教師あり学習では、入力$x$と出力$y$を

\subsection{機械学習モデル}
MLP
\subsubsection{パーセプトロン}
\begin{figure}[tb]
  \centering
  \rule{8cm}{6cm}
  \caption{パーセプトロン}
  \label{fig:fit_def}
\end{figure}

\subsubsection{ニューラルネットワーク}
ニューラルネットワークは入力に対し、線形変換と非線形変換を繰り返し適応することで、入力から予測値を計算することができる。
\begin{figure}[tb]
  \centering
  \includegraphics[clip, width=10cm]{fig/4/MLP.png}
  \caption{MLP}
  \label{fig:Resolution}
\end{figure}

\subsubsection{・活性化関数}
\subsubsection{・損失関数}

\section{機械学習を用いた CW 作成}
\subsection{入力データに対する事前処理}
・磁場構造に考慮した領域分け<学習領域の図>\\
\begin{figure}[tb]
  \centering
  \includegraphics[clip, width=7cm]{fig/4/c1_withMag.pdf}
  \caption{磁場構造に考慮するための学習領域の分け方}
  \label{fig:Mag}
\end{figure}

・ヒットマップクリーナ<クリーナー前後のhitmap>
\begin{figure}[tb]
  \centering
  \includegraphics[clip, width=14cm]{fig/4/cleaner.png}
  \caption{ヒットマップクリーナーをかけた前後のミューオンヒットマップの例}
  \label{fig:hitmapcleaner}
\end{figure}


\subsection{機械学習モデルの設計方法とトレーニング}
・Tensorflow\\
\subsubsection{機械学習モデルの設計}

\subsubsection{トレーニング}

\subsubsection{機械学習モデルの性能評価}
\begin{figure}[tb]
  \centering
  \rule{8cm}{6cm}
  %\includegraphics[clip, width=14cm]{}
  \caption{残差分布(MC)}
  \label{fig:fit_def}
\end{figure}

\begin{figure}[tb]
  \centering
  \rule{8cm}{6cm}
  %\includegraphics[clip, width=14cm]{}
  \caption{trueに対するpredの分布(MC)}
  \label{fig:fit_def}
\end{figure}

\begin{figure}[tb]
  \centering
  \rule{8cm}{6cm}
  %\includegraphics[clip, width=14cm]{}
  \caption{true-prepのpt分布(MC)}
  \label{fig:fit_def}
\end{figure}

\begin{figure}[tb]
  \centering
  \rule{8cm}{6cm}
  %\includegraphics[clip, width=14cm]{}
  \caption{残差分布(Data)}
  \label{fig:fit_def}
\end{figure}

\begin{figure}[tb]
  \centering
  \rule{8cm}{6cm}
  %\includegraphics[clip, width=14cm]{}
  \caption{trueに対するpredの分布(Data)}
  \label{fig:fit_def}
\end{figure}

\begin{figure}[tb]
  \centering
  \rule{8cm}{6cm}
  %\includegraphics[clip, width=14cm]{}
  \caption{true-prepのpt分布(Data)}
  \label{fig:fit_def}
\end{figure}


\subsection{出力データから CW の作成}
\subsubsection{出力データ}
\subsubsection{fitting}


4bit


・Efficiencyに対するfitting <fittingの定義のプロット><threshould><Resolution><プラトー>\\
\begin{figure}[tb]
  \centering
  \includegraphics[clip, width=14cm]{fig/4/fitting_def.png}
  \caption{fittingの定義}
  \label{fig:fit_def}
\end{figure}

\begin{figure}[tb]
  \centering
  \includegraphics[clip, width=14cm]{fig/4/resolution_v07_v05.png}
  \caption{Resolution}
  \label{fig:Resolution}
\end{figure}

\subsubsection{出力データをPt閾値に変換}

\subsection{作成した CW の評価}
Run-2データを用いた評価

<15段階の閾値のEfficiency>
\begin{figure}[tb]
  \centering
  \includegraphics[clip, width=14cm]{fig/4/v07_15_Eff.png}
  \caption{Resolution}
  \label{fig:Resolution}
\end{figure}


<15段階それぞれのEfficiency(v05 vs v06,v07)>\\
\begin{figure}[tb]
  \centering
  \includegraphics[clip, width=14cm]{fig/4/hikaku_v05_v06.png}
  \caption{v05v06}
  \label{fig:Resolution}
\end{figure}

\begin{figure}[tb]
  \centering
  \includegraphics[clip, width=14cm]{fig/4/hikaku_v05_v07.png}
  \caption{v05v07}
  \label{fig:Resolution}
\end{figure}

<MCとDataの比較>
\begin{figure}[tb]
  \centering
  \includegraphics[clip, width=14cm]{fig/4/hikaku_v06_v07.png}
  \caption{v06v07}
  \label{fig:Resolution}
\end{figure}

<TriggerRate>
\begin{figure}[tb]
  \centering
  \includegraphics[clip, width=14cm]{fig/4/rate_v05_v06.png}
  \caption{v05v06}
  \label{fig:Resolution}
\end{figure}

