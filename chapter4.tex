\chapter{機械学習によるCWの作成手法}
本章では機械学習を用いてCWを作成する手法について述べる。
本研究では、シミュレーション用のCWと実際の測定用のCWをそれぞれ作成している。
また、作成したCWの性能について、それぞれ評価を行う。

\section{機械学習}
機械学習とは、データからコンピューターが自動で特徴量やルールを学習し、学習した結果に基づいて新たなデータに対し分類や予測を行う分析手法の一つである。

機械学習には、教師あり学習、教師なし学習、強化学習

・回帰問題\\
・分類問題
\subsection{パーセプトロン}

\subsection{ニューラルネットワーク}
\subsubsection{活性化関数}
\subsubsection{損失関数}

\section{機械学習を用いたCWの作成}
\subsection{入力データに対する事前処理}
・磁場構造に考慮した領域分け\\
・ヒットマップクリーナ
\subsection{機械学習モデル}
・Tensorflow\\
・モデルのトレーニング\\
・パラメータ
\subsection{ミューオン運動量の閾値の決定}
・L1-Efficiency\\
・0.1GeV刻みのEfficiency\\
・fitting\\
・threshould-ptの定義
\subsection{性能評価}
・Efficiency(v05 vs v06,v07)\\
・TriggerRate
