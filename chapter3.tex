\chapter{初段ミューオントリガーシステム}\label{chapter3}
ATLAS 実験におけるミューオントリガーは、RPC を用いるバレル部と TGC を用いるエンドキャップ部に分かれている。
本章では、Run-3 におけるエンドキャプ部初段ミューオントリガーシステムについて述べる。

\section{エンドキャプ部初段ミューオントリガー}
ATLAS 実験におけるミューオントリガーに用いる検出器は、図~\ref{fig:muon}に示すようにRPCを用いるバレル部とTGCを用いるエンドキャップ部に分かれている。以下ではTGCを用いるエンドキャップ部でのトリガーシステムについて説明する。エンドキャプ部はさらに2つの領域に分けられ、$1.05 < |\eta| < 1.9$をエンドキャップ領域、$1.9 < |\eta| < 2.4$をフォワード領域と呼ぶ。
\begin{figure}[tb]
  \centering
  \includegraphics[clip, width=14cm]{fig/2/ch01_fig_03a.pdf}
  \caption{初段ミューオントリガーに用いる検出器の設置位置\cite{article:phase2}。バレル部のトリガー判定に用いるRPCと、エンドキャップ部のトリガー判定に用いるTGCがATLAS検出器の最外層に設置されている。}
  \label{fig:muon}
\end{figure}

\subsection{ミューオントリガーの概要}\label{section:CW}
エンドキャップ部の初段ミューオントリガーで用いられるトリガー判定の概要を図~\ref{fig:trigger-scheme}に示す。
衝突点で生成されたミューオンはトロイド磁石の磁場領域より内側にある検出器を通過した後、トロイド磁場領域を通りTGCに到達する。トロイド磁石による磁場は$\phi$方向にかかっているため、ミューオンの飛跡はトロイド磁場中で$\eta$方向に曲げられる。さらに、衝突点付近のソレノイド磁石で生じるz方向の磁場成分と、トロイド磁石付近で生じたR方向の磁場成分によって、ミューオンの飛跡は$\phi$方向にも曲げられる。ミューオンの飛跡の曲がり具合は横方向運動量$p_T$の大きさによって変化するため、測定した飛跡情報からミューオンの$p_T$を算出することができ、トリガー判定に使用することができる。

トロイド磁場によって曲げられたミューオンは TGC-BW の M1, M2, M3 を通過し、ヒット情報から飛跡を再構成される。また、衝突点と M3 のヒット位置を結んだ直線をミューオンが無限運動量で通過したと仮定した場合の飛跡として扱う。この無限運動量を持つミューオンの飛跡と磁場によって曲げられた実際の飛跡を比較し、M1 におけるヒット位置の$R$方向と$\phi$方向のずれ($dR$, $d\phi$)を計算する。この($dR$, $d\phi$)は磁場による飛跡の曲がり具合を表している。
この$dR$と$d\phi$の値が大きいミューオンは、磁場中で大きく曲げられたことを意味しており、小さい$p_T$として判定される。逆に$dR$と $d\phi$ の値がが小さいほど、磁場中ではあまり曲がらないミューオンの飛跡であるため、大きな $p_T$ として判定される。
そして、事前に($dR$, $d\phi$)に対応する$p_T$の関係をLook-Up Table~(LUT)として保存しておき、トリガー判定の際にLUTを参照する事で短時間での$p_T$の出力を可能としている。

\begin{figure}[tb]
  \centering
  \includegraphics[clip, width=15cm]{fig/3/akatsuka_mt_trigger_scheme.pdf}
  \caption{ATLAS検出器エンドキャップ領域におけるトリガースキームの概念図\cite{article:akatsuka-mron}。無限大の運動量を持つミューオンを仮定し、磁場によって曲げられたミューオンとの位置の差($dR$, $d\phi$)を用いて$p_T$を計算する。}
  \label{fig:trigger-scheme}
\end{figure}

LUTとは入力データに対応する出力データを参照するための表のことを指し、初段ミューオントリガーでは、($dR$, $d\phi$)を入力として$p_T$を出力するLUTをFPGAに保存している。

\subsubsection{Coincidence Window}
このLUTはCoincidence Window~(CW)と呼ばれており、初段ミューオントリガーでは事前にシミュレーションデータを用いて($dR$, $d\phi$)に対応する$p_T$を算出し、Run-3では図~\ref{fig:CW}のように15段階の$p_T$閾値を判定できるCWをFPGAに保存している。

Run-3で使用されるCWは図~\ref{fig:CW}の色のように15個に分類されている。この各色が15段階の$p_T$閾値に対応しており、マスの中の数字は表~\ref{pt_number}に示す$p_T$ numberと対応している。また、各$p_T$閾値の符号はミューオンの電荷に対応している。図~\ref{fig:CW}の
エンドキャップ部のトロイド磁場やTGCは理想的には8回対称であるが、磁場の向きやTGCチェンバーの設置位置のズレなどがあるため、CWはTGC-BWのトリガー判定に用いられる単位位置情報ごとに独立に作成されている。

\begin{figure}[tb]
  \centering
  \includegraphics[clip, width=7cm]{fig/3/cw_run3_shiomi.png}
  \caption{Run-3でのTGCにおけるCoincidence Windowの例\cite{article:shiomi-mron}。ミューオンのヒットがあった時にそれぞれの検出位置ののCWを参照し、($dR$, $dφ$)からミューオンの$p_T$を15段階で見積もる。}
  \label{fig:CW}
\end{figure}

\begin{table}[]
    \caption{Run-3初段ミューオントリガーにおける15段階の$p_T$閾値\cite{article:shiomi-mron}。}
    \label{pt_number}
    \centering
    %\begin{tabular}{|c|c|c|c|c|c|c|c|c|c|c|c|c|c|c|c|c|c|c|c|c|c|c|c|}
    \begin{tabular}{|c|c|c|c|}
        \hline
        $p_T$ number & Threshould name & Status\\
        \hline
        1 & L1$\_$MU3 & $p_T \geq$ 3 GeV \\
        \hline
        2 & L1$\_$MU4 & $p_T \geq$ 4 GeV \\
        \hline
        3 & L1$\_$MU5 & $p_T \geq$ 5 GeV \\
        \hline
        4 & L1$\_$MU6 & $p_T \geq$ 6 GeV \\
        \hline
        5 & L1$\_$MU7 & $p_T \geq$ 7 GeV \\
        \hline
        6 & L1$\_$MU8 & $p_T \geq$ 8 GeV \\
        \hline
        7 & L1$\_$MU9 & $p_T \geq$ 9 GeV \\
        \hline
        8 & L1$\_$MU10 & $p_T \geq$ 10 GeV \\
        \hline
        9 & L1$\_$MU11 & $p_T \geq$ 11 GeV \\
        \hline
        10 & L1$\_$MU12 & $p_T \geq$ 12 GeV \\
        \hline
        11 & L1$\_$MU13 & $p_T \geq$ 13 GeV \\
        \hline
        12 & L1$\_$MU14 & $p_T \geq$ 14 GeV \\
        \hline
        13 & L1$\_$MU15 & $p_T \geq$ 15 GeV \\
        \hline
        14 & L1$\_$MU18 & $p_T \geq$ 18 GeV \\
        \hline
        15 & L1$\_$MU20 & $p_T \geq$ 20 GeV \\
        \hline
        %$p_t$ number & 1 & 2 & 3 & 4 & 5 & 6 & 7 & 8 & 9 & 10 & 11 & 12 & 13 & 14 & 15\\
        %\hline
        %$p_T$ Threshould[GeV] & 3 & 4 & 5 & 6 & 7 & 8 & 9 & 10 & 11 & 12 & 13 & 14 & 15 & 18 & 20\\
        %$p_T$ Threshould & 3 & 4 & 5 & 6 & 7 & 8 & 9 & 10 & 11 & 12 & 13 & 14 & 15 & 18 & 20\\
        %\hline
    \end{tabular}
\end{table}

また、CWはコインシデンスのタイプによって4種類存在する。
一つ目はM1からM3までの3つのステーション(M1, M2, M3)全てにおいてワイヤーとストリップともにヒットが確認された3-3ステーションコインシデンス、二つ目がワイヤーは3ステーションにヒットが確認されたが、ストリップに関しては2ステーション(M2, M3)のヒットしか確認できなかった3-2ステーションコインシデンス、三つ目がワイヤーは2ステーションしかヒットが確認されなかったが、ストリップでは3ステーションにヒットが確認された2-3 ステーションコインシデンス、四つ目がワイヤーとストリップともに2ステーションしかヒットが確認できなかった2-2ステーションコインシデンスである。
%3ステーションコインシデンスフラグはストリップ、ワイヤー共に3ステーションでヒットがあった場合に立つフラグである。
2ステーションコインシデンスか 3ステーションコインシデンスかでズレを計算する検出器がM2-M3か、M1-M3か変わってくるため、(dR, dφ)の範囲も変わってくる。2ステーションの場合、$−7 \geq dR \geq 7$、$−3 \geq d\phi \geq 3$、3 ステーションの場合、$−15 \geq dR \geq 15$、$−7 \geq d\phi \geq 7$ の範囲で定義される。


\subsubsection{トリガーセクター}
TGC のトリガー判定に用いられる単位の模式図を図~\ref{fig:RoI}に示す。
図\ref{fig:RoI} に示すように、TGCの検出領域は$\phi$方向にエンドキャプ領域では48個、フォワード領域では24個に分割しており、トリガー回路的に独立していて "トリガーセクター" と呼ばれる。このトリガーセクターはさらに小さな領域である Region of Interest (RoI) に分割され、エンドキャップ領域のトリガーセクターは $\eta$ 方向に 37 分割、$\phi$ 方向に 4 分割されるため 148 個の RoI で構成されている。フォワード領域のトリガーセクターは $\eta$ 方向に 16 分割、$\phi$ 方向に 4 分割されるため 64 個の RoI で構成されている。また RoI を $\eta$ 方向に 2 つ、$\phi$ 方向に 4 つまとめたものを Sub Sector Cluster (SSC) と呼ぶ。RoI は TGC の持つミューオンの検出位置情報の最小単位である。$p_T$ 判定に用いる CWはこのRoIごとに作成しているため、TGCのA-sideとC-sideで合計$(148\times48+64\times24)\times=17280$個のCWがトリガー判定には必要である。

\begin{figure}[tb]
  \centering
  \includegraphics[clip, width=13cm]{fig/3/RoI.png}
  \caption{TGC におけるトリガーセクターと RoI の模式図。緑の線で囲まれた領域が 1 つのトリガーセクターを表し、赤の線で囲まれたマスが 1 つの RoI を表す。}
  \label{fig:RoI}
\end{figure}

\subsubsection{インナーコインシデンス}
エンドキャップ部のミューオントリガーには陽子衝突由来でない荷電粒子により発行されたトリガー~(フェイクトリガー)が存在し、レートを上げる要因になっている。Run-2ではTGCのヒット情報に対してTGC-EI/FIとTileカロリメータの情報を使ったコインシデンス~(インナーコインシデンス) を取ることでフェイクトリガーを大きく削減することができた。しかし、$1.9 < |\eta| < 2.4$ の領域ではインナーコインシデンスをとるためのトリガー用検出器がないためフェイクトリガーが多く残っている。
Run-3からはNSWとRPC~BIS78の導入により、Run-2より広範囲でインナーコインシデンスをとることが可能になるため、フェイクトリガーをより削減できる。
Tileカロリメータとコインシデンスだけでなく、新たに導入されるNSWやRPC~BIS78をインナーコインシデンスに用いた場合に期待されるトリガー発行数の分布を図\ref{fig:Rate_innercoincidence}に示す。

\begin{figure}[tb]
  \centering
    \includegraphics[clip, width=14cm]{fig/3/ATL-COM-DAQ-2018-033-fig2.pdf}
  \caption{Run-3 で期待される $p_T$ 閾値 20 GeV におけるトリガー発行数の $\eta$ 分布。白色、水色、黄色の領域はそれぞれ Tile カロリメータ、RPC BIS78、NSW を用いたインナーコインシデンスを導入した場合に削減できるトリガー発行数を示す。青色の領域は Run-3で期待されるトリガー発行数、赤色の領域は発行されたトリガーのうちオフラインで再構成されるミューオンの数を示す。緑の分布はオフラインで再構成されたミューオンのうち、$p_T$ が 20 GeV 以上のミューオンの数を示す。}
  \label{fig:Rate_innercoincidence}
\end{figure}

\subsection{初段ミューオントリガーにおけるエレクトロニクス}
初段ミューオントリガーでは、ATLAS 検出器から送られてくる情報に対して SectorLogic、MUCTPI、L1Topo、CTP という電子回路を経てトリガーが発行される。
以下では各エレクトロニクスについて説明する。

\subsubsection{Amplifier Shaper Discriminator (ASD) ボード}
Amplifier Shaper Discriminator (ASD) ボードは TGC のワイヤーとストリップからアナログ信号を受け取り、デジタル信号への変換を行う。
ASD ボード上の ASD において TGC からのアナログ信号を増幅・整形し、 閾値電圧を超えた信号のみ LVDS 信号として出力される。1 枚の ASD ボードは 4 つの ASD ASIC を搭載しており、ASD ASIC は 4 つの信号の受信・処理を行う。そのため、同時に 16 チャンネルの信号を処理することが可能である。

\subsubsection{Patch-Panel ASIC (PP ASIC)}
Patch-Panel ASIC は ASD からワイヤーとストリップそれぞれの LVDS 信号を受け取り、タイミングの調整を行うことで、同じ陽子衝突由来の信号を同時に次の SLB ASIC に送る。陽子衝突が起きてからミューオンが検出器に到達する時間や、ケーブルの長さの違いにより、信号のタイミングが各チャンネルごとに異なるため、PP ASIC を用いてタイミングの調整を行う。

\subsubsection{Slave Board ASIC (SLB ASIC)}
Slave Board ASIC は読み出しとトリガー判定の 2 種類の処理を行う。
トリガー判定で行う処理としては、各チャンネルの情報を用いてコインシデンスを取ることである。
TGC Triplet (M1 ステーション) ではワイヤーの場合は 3 層中 2 層にヒットがあることを要求し、ストリップの場合は 2 層中 1 層にヒットがあることを要求する。
2 つの TGC Doublet (M2、M3 ステーション) では、各ステーションから信号を受け取りワイヤーとストリップで独立に 4 層中 3 層以上にヒットがあることを要求する。 これらのコインシデンス結果はLVDS 信号で後段の High PT ボードに送る。

\subsubsection{High PT (HPT) ボード}
High PT ボードは、M1 の SLB と M2,M3 の SLB からのコインシデンス結果を受け取り、 M1,M2,M3 の 3つのステーション間のコインシデンスを行う。M1 と M3 の位置情報から ($\Delta R$, $\Delta \phi$) を計算し、次の Sector Logic に送る。Sector Logic にはボードごとに、位置情報 $R$ と $\phi$ 、位置の差の情報 $\Delta R$ と $\Delta \phi$ を G-Link 通信を用いて送信する。データ通信速度の制限により、1 つの HPT ASIC から最大 2 候補を選んで送信している。

\subsubsection{New Sector Logic (NSL)}
New Sector Logic では TGC-BW とトロイド磁石の内側にある検出器の情報を統合してトリガー判定を行う。
TGC-BW の HPT ボードからは、ミューオン候補の情報が G-Link 規格で送られてくる。
RPC BIS78、Tile カロリメータ、 TGC-EI からは、検出器におけるヒット情報が送られてくる。
NSW からは通過したミューオンの飛跡情報が送られてくる。
NSL はこれらの情報をもとにトリガー判定を行い、トリガー判定の結果を MUCTPI に送信する。

NSL ボードでは、 HPT ボードから受け取った TGC BW の位置情報 $R$ と $\phi$ 、位置の差の情報 $\Delta R$ と $\Delta \phi$ を用いて $p_T$ の判定を行う。各 ($R$, $\phi$) からミューオンのヒット位置を表す RoI を決定し、($\Delta R$, $\Delta \phi$) から NSL 上に実装されている Coincidence Window を用いて $p_T$ に変換する。




\section{CW の作成及び最適化手法}\label{section:最適化}
\subsection{Run-3 に向けた Coincidence Window の作成手法}
Run-3 に向けた 15 段階の $p_T$ 閾値に対応した Coincidence Window (CW) は先行研究で作成されており、本節ではこの作成手法について述べる。

全ての RoI に対して pT がある閾値よりも高いミューオンについて dR、d$\phi$ の二次元分布図を作成する。これをヒットマップと呼び、閾値が 1 GeV から 40 GeV まで 1 GeV 刻みにヒットマップを作成する。ヒットマップ作成のために 1500 万イベントのシングルミューオンの MCサンプルを使用している。
作成したヒットマップにはミューオンの多重散乱などが原因で、孤立している部分、穴の空いている部分、偶然ヒットした部分が存在する。そこで、ヒットマップのエントリー数などの条件をかけることでヒットマップの整形を行う。そして、1 GeV から 40 GeV までのヒットマップを重ね合わせることで 40 段階の CW を作成する。
その後、別のシングルミューオンの MC サンプルを使用し、40 段階の各領域での $p_T$ 分布を作成する。この $p_T$ 分布に対しガウス分布でフィッティングしたときの Mean 値をその領域の $p_T$ として設定し、そこから 15 段階の $p_T$ 閾値を選択することで Run-3 に対応した CW を作成する。



\subsection{Run-2 における CW の最適化手法}
TGC 検出器は検出器の入れ替えなどのために検出器の移動を行うことがあり、検出器のズレが生じてしまう。TGC の設置位置のズレの測定方法はこれまでの研究で既に確立されている。
TGC 検出器のズレを示すパラメータを図\ref{fig:dr_para}のように定義し、図\ref{fig:ズレ}に Run-2 での実データを用いて測定した TGC 検出器の設置位置のズレを示す。
\begin{figure}[tb]
  \centering
  \includegraphics[clip, width=10cm]{fig/3/drho_param_position_measurement.png}
  \caption{TGC 検出器のズレ$\delta\rho$を表すパラメータの定義。x軸方向のズレを$\delta r$、z軸方向のズレを$-dz\tan\theta$と表し、全体としてのズレを$\delta\rho = \delta r -dz\tan\theta$と定義する。}
  \label{fig:dr_para}
\end{figure}

\begin{figure}
    %\begin{tabular}{cc}
    \begin{minipage}[tb]{0.4\linewidth}
        \centering
        \includegraphics[clip, width=7cm]{fig/3/TGCAlign_CW.muon.bias.20160606.v1.A-side.pdf}
        \vspace{10pt}
        \subcaption{}
        \label{}
    \end{minipage}
    \hfill
    \begin{minipage}[tb]{0.4\linewidth}
        \centering
        \includegraphics[clip, width=7cm]{fig/3/TGCAlign_CW.muon.bias.20160606.v1.C-side.pdf}
        \vspace{10pt}
        \subcaption{}
        \label{}
    \end{minipage}
    \caption{Run-2での検出器のズレの測定図。(a):A-side、(b):C-side}
    \label{fig:ズレ}
    %\end{tabular}
\end{figure}

CW の作成には TGC が設計通りの位置に設置されているシミュレーションデータを使用するため、作成された CW には実際の TGC の設置位置によるズレが考慮されていない。そのため、実際の測定においてミューオンの飛跡が想定されたものから外れてしまい、$p_T$ 判定を行うミューオントリガーの運動量分解能の低下を招く原因となる。

また、ミューオンの電荷にも影響がある。
\begin{figure}[tb]
  \centering
  \includegraphics[clip, width=7cm]{fig/3/charge_14gev_phi2_eta11.pdf}
  \caption{あるチェンバーにおける電荷別のTurn-on curveの例。赤が正電荷、青が負電荷のTurn-on curveである。}
  \label{fig:fit_def}
\end{figure}

そこで、CW に対して検出器のズレを補正し最適化することで、トリガー効率を維持しつつ、トリガーレート削減を目指す研究が行われていた。
以下では、Run-2 で行われた TGC 検出器の設置位置のズレや歪みを補正することで CW を最適化する方法について述べる。


RoI 毎に pT 閾値を境目としたミューオン分布を用いた CW の cell 毎に判定するパラメータ $x$ を定義する。
\begin{equation}
    x = \frac{N_{p_{T}>20GeV}}{\sqrt{N^2_{p_{T}>20GeV}+N^2_{p_{T}<20GeV}}}
 \label{equ:fitting}
\end{equation}


・検出器アライメント
<山内さんの手法>
<木戸さんの手法の説明>


\section{本研究の目的}
CW は事前にシミュレーションデータを用いたミューオンの運動量分布から統計的に作成する。しかし、実際の検出器にはシミュレーション上にはないズレや歪みが存在しており、シミュレーションデータから作成した CW をそのまま適応すると、トリガー性能の悪化に繋がってしまう。そこで、Run-2 では実際のデータをもとに、ある運動量を持つミューオンのヒットマップを作成し、ヒットマップと CW を比較することで検出器アライメントの補正値を判断して CW の最適化を行っていた。
このように従来の CW の作成から最適化までの作業では、大量のシミュレーションデータや実際のデータの傾向を CW に反映させる作業を手動で行っていた。
初段ミューオントリガーのアップグレードに伴って、Run-3 に向けた CW を作り直す必要があり、Run-3 においてもトリガー性能を向上させるためには Run-2 と同様に CW の最適化を行う必要がある。しかし、\ref{section:最適化}節で述べた Run-2 で行われていた CW の作成及び最適化手法は膨大な作業量であり、Run-3 に対応した CW に向けて Run-2 と同様の作成及び最適化手法を行うことは現実的でない。
そこで、本研究では近年急速に発展している機械学習に着目し、新たな CW の最適化手法の開発を行う。















