\chapter{初段ミューオントリガーシステム}\label{chapter3}
ATLAS実験における初段ミューオントリガーは、RPCを用いるバレル部とTGCを用いるエンドキャップ部に分かれている。
本章では、Run-3におけるエンドキャプ部の初段ミューオントリガーシステムについて述べる。

\section{エンドキャプ部初段ミューオントリガー}

\subsection{トリガーアルゴリズムの概要}\label{section:CW}

\subsubsection{ミューオンの横方向運動量の判定}

\subsubsection{インナーコインシデンス}\label{innnercoin}

\subsubsection{トリガー判定に用いられる位置情報}

%\newpage
\subsection{初段ミューオントリガーにおけるエレクトロニクス}

\subsubsection{Amplifier Shaper Discriminatorボード}

\subsubsection{Patch-Panel ASIC}

\subsubsection{Slave Board ASIC}

\subsubsection{High PT ボード}

\subsubsection{New Sector Logic}

\section{従来のCWの作成及び最適化手法}\label{section:最適化}
\subsection{従来のCWの作成方法}

\subsection{CWの最適化手法}
\subsubsection{TGCの設置位置の測定}\label{ズレ}


\subsubsection{Run-2における最適化手法}

\section{本研究の目的}