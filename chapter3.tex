\chapter{エンドキャプ部初段ミューオントリガーシステム}
前章ではATLAS実験のトリガーシステムについて述べた。このトリガーシステムの一つである初段ミューオントリガーシステムに着目し、本章では、Run-3におけるエンドキャプ部初段ミューオントリガーシステムについて述べる。

\section{エンドキャプ部初段ミューオントリガー}
\subsection{Thin Gap Chamber (TGC)}

\subsection{トリガーセクター}
ROIの話
\subsection{ミューオンの運動量の算出}

\subsubsection{Coincidence Window (CW)}
CWの話

\subsection{エレクトロニクス}



\section{現行のエンドキャプ部初段ミューオントリガー}
・現行CWの作成手法
・Efficiency
・Rate

\section{本研究の目的}
・検出器のズレ
\subsection{Run-2におけるCWの作成及び最適化手法}
・作成
・検出器アライメント
\subsection{本研究の方針}
Run-3における初段ミューオントリガーの性能を向上させるために、現状では上記で述べた最適化手法を行う必要がある。しかし、Run-2で行われていた最適化手法は、各CWごとに1マスずつ6段階の閾値を確認する手法のため膨大な作業量が必用である。


・機械学習に注目



\begin{figure}[tb]
  \centering
  \includegraphics[clip, width=14cm]{fig/3/akatsuka_mt_trigger_scheme.pdf}
  \caption{ATLAS検出器エンドキャップ領域におけるトリガースキームの概念図\cite{article:akatsuka-mron}。無限大の運動量を持つミューオンを仮定し、磁場によって曲げられたミューオンとの位置の差を用いて$\pt$を計算する。}
  \label{fig:trigger-scheme}
\end{figure}









